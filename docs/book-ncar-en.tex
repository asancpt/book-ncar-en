\documentclass[12pt,]{krantz}
\usepackage{lmodern}
\usepackage{amssymb,amsmath}
\usepackage{ifxetex,ifluatex}
\usepackage{fixltx2e} % provides \textsubscript
\ifnum 0\ifxetex 1\fi\ifluatex 1\fi=0 % if pdftex
  \usepackage[T1]{fontenc}
  \usepackage[utf8]{inputenc}
\else % if luatex or xelatex
  \ifxetex
    \usepackage{mathspec}
  \else
    \usepackage{fontspec}
  \fi
  \defaultfontfeatures{Ligatures=TeX,Scale=MatchLowercase}
    \setmonofont[Mapping=tex-ansi,Scale=0.7]{Source Code Pro}
\fi
% use upquote if available, for straight quotes in verbatim environments
\IfFileExists{upquote.sty}{\usepackage{upquote}}{}
% use microtype if available
\IfFileExists{microtype.sty}{%
\usepackage[]{microtype}
\UseMicrotypeSet[protrusion]{basicmath} % disable protrusion for tt fonts
}{}
\PassOptionsToPackage{hyphens}{url} % url is loaded by hyperref
\usepackage[unicode=true]{hyperref}
\PassOptionsToPackage{usenames,dvipsnames}{color} % color is loaded by hyperref
\hypersetup{
            pdftitle={R packages for Noncompartmental Analysis},
            pdfauthor={Sungpil Han},
            colorlinks=true,
            linkcolor=Maroon,
            citecolor=Blue,
            urlcolor=Blue,
            breaklinks=true}
\urlstyle{same}  % don't use monospace font for urls
\usepackage{natbib}
\bibliographystyle{apalike}
\usepackage{color}
\usepackage{fancyvrb}
\newcommand{\VerbBar}{|}
\newcommand{\VERB}{\Verb[commandchars=\\\{\}]}
\DefineVerbatimEnvironment{Highlighting}{Verbatim}{commandchars=\\\{\}}
% Add ',fontsize=\small' for more characters per line
\usepackage{framed}
\definecolor{shadecolor}{RGB}{248,248,248}
\newenvironment{Shaded}{\begin{snugshade}}{\end{snugshade}}
\newcommand{\KeywordTok}[1]{\textcolor[rgb]{0.13,0.29,0.53}{\textbf{#1}}}
\newcommand{\DataTypeTok}[1]{\textcolor[rgb]{0.13,0.29,0.53}{#1}}
\newcommand{\DecValTok}[1]{\textcolor[rgb]{0.00,0.00,0.81}{#1}}
\newcommand{\BaseNTok}[1]{\textcolor[rgb]{0.00,0.00,0.81}{#1}}
\newcommand{\FloatTok}[1]{\textcolor[rgb]{0.00,0.00,0.81}{#1}}
\newcommand{\ConstantTok}[1]{\textcolor[rgb]{0.00,0.00,0.00}{#1}}
\newcommand{\CharTok}[1]{\textcolor[rgb]{0.31,0.60,0.02}{#1}}
\newcommand{\SpecialCharTok}[1]{\textcolor[rgb]{0.00,0.00,0.00}{#1}}
\newcommand{\StringTok}[1]{\textcolor[rgb]{0.31,0.60,0.02}{#1}}
\newcommand{\VerbatimStringTok}[1]{\textcolor[rgb]{0.31,0.60,0.02}{#1}}
\newcommand{\SpecialStringTok}[1]{\textcolor[rgb]{0.31,0.60,0.02}{#1}}
\newcommand{\ImportTok}[1]{#1}
\newcommand{\CommentTok}[1]{\textcolor[rgb]{0.56,0.35,0.01}{\textit{#1}}}
\newcommand{\DocumentationTok}[1]{\textcolor[rgb]{0.56,0.35,0.01}{\textbf{\textit{#1}}}}
\newcommand{\AnnotationTok}[1]{\textcolor[rgb]{0.56,0.35,0.01}{\textbf{\textit{#1}}}}
\newcommand{\CommentVarTok}[1]{\textcolor[rgb]{0.56,0.35,0.01}{\textbf{\textit{#1}}}}
\newcommand{\OtherTok}[1]{\textcolor[rgb]{0.56,0.35,0.01}{#1}}
\newcommand{\FunctionTok}[1]{\textcolor[rgb]{0.00,0.00,0.00}{#1}}
\newcommand{\VariableTok}[1]{\textcolor[rgb]{0.00,0.00,0.00}{#1}}
\newcommand{\ControlFlowTok}[1]{\textcolor[rgb]{0.13,0.29,0.53}{\textbf{#1}}}
\newcommand{\OperatorTok}[1]{\textcolor[rgb]{0.81,0.36,0.00}{\textbf{#1}}}
\newcommand{\BuiltInTok}[1]{#1}
\newcommand{\ExtensionTok}[1]{#1}
\newcommand{\PreprocessorTok}[1]{\textcolor[rgb]{0.56,0.35,0.01}{\textit{#1}}}
\newcommand{\AttributeTok}[1]{\textcolor[rgb]{0.77,0.63,0.00}{#1}}
\newcommand{\RegionMarkerTok}[1]{#1}
\newcommand{\InformationTok}[1]{\textcolor[rgb]{0.56,0.35,0.01}{\textbf{\textit{#1}}}}
\newcommand{\WarningTok}[1]{\textcolor[rgb]{0.56,0.35,0.01}{\textbf{\textit{#1}}}}
\newcommand{\AlertTok}[1]{\textcolor[rgb]{0.94,0.16,0.16}{#1}}
\newcommand{\ErrorTok}[1]{\textcolor[rgb]{0.64,0.00,0.00}{\textbf{#1}}}
\newcommand{\NormalTok}[1]{#1}
\usepackage{longtable,booktabs}
% Fix footnotes in tables (requires footnote package)
\IfFileExists{footnote.sty}{\usepackage{footnote}\makesavenoteenv{long table}}{}
\IfFileExists{parskip.sty}{%
\usepackage{parskip}
}{% else
\setlength{\parindent}{0pt}
\setlength{\parskip}{6pt plus 2pt minus 1pt}
}
\setlength{\emergencystretch}{3em}  % prevent overfull lines
\providecommand{\tightlist}{%
  \setlength{\itemsep}{0pt}\setlength{\parskip}{0pt}}
\setcounter{secnumdepth}{5}
% Redefines (sub)paragraphs to behave more like sections
\ifx\paragraph\undefined\else
\let\oldparagraph\paragraph
\renewcommand{\paragraph}[1]{\oldparagraph{#1}\mbox{}}
\fi
\ifx\subparagraph\undefined\else
\let\oldsubparagraph\subparagraph
\renewcommand{\subparagraph}[1]{\oldsubparagraph{#1}\mbox{}}
\fi

% set default figure placement to htbp
\makeatletter
\def\fps@figure{htbp}
\makeatother


\title{R packages for Noncompartmental Analysis}
\author{Sungpil Han}
\date{2018-03-08}

\usepackage{amsthm}
\newtheorem{theorem}{Theorem}[chapter]
\newtheorem{lemma}{Lemma}[chapter]
\theoremstyle{definition}
\newtheorem{definition}{Definition}[chapter]
\newtheorem{corollary}{Corollary}[chapter]
\newtheorem{proposition}{Proposition}[chapter]
\theoremstyle{definition}
\newtheorem{example}{Example}[chapter]
\theoremstyle{definition}
\newtheorem{exercise}{Exercise}[chapter]
\theoremstyle{remark}
\newtheorem*{remark}{Remark}
\newtheorem*{solution}{Solution}
\begin{document}
\maketitle

{
\hypersetup{linkcolor=black}
\setcounter{tocdepth}{2}
\tableofcontents
}
\listoftables
\listoffigures
\chapter*{Overview}\label{overview}


Noncompartmental analysis (NCA) is a primary analytical approach for
pharmacokinetic studies, and its parameters act as decision criteria in
bioequivalent studies. Currently, NCA is usually carried out by
commercial softwares such as WinNonlin®. In this article, we introduce
our newly-developed two R packages, NonCompart (NonCompartmental
analysis for pharmacokinetic data) and ncar (NonCompartmental Analysis
for pharmacokinetic Report), which can perform NCA and produce complete
NCA reports in both pdf and rtf formats. These packages are compatible
with CDISC (Clinical Data Interchange Standards Consortium) standard as
well. We demonstrate how the results of WinNonlin® are reproduced and
how NCA reports can be obtained. With these R packages, we aimed to help
researchers carry out NCA and utilize the output for early stages of
drug development process. These R packages are freely available for
download from the CRAN repository.

\chapter*{Acknowledgements}\label{acknowledgements}


This research was supported by the EDISON (EDucation- research
Integration through Simulation On the Net) Program through the National
Research Foundation of Korea (NRF) funded by the Ministry of Education,
Science and Technology (Grant number: 016M3C1A6936614). We thank
Dr.~Joon Seo Lim from the Scientific Publications Team at Asan Medical
Center for his editorial assistance in preparing this manuscript.

\mainmatter

\chapter{Introduction}\label{introduction}

The aim of pharmacokinetics (PK) studies is to examine the kinetics of a
drug with regard to absorption, distribution, metabolism and elimination
in the body. PK data analysis consists of noncompartmental analysis
(NCA) and nonlinear regression analysis. \citep{acharya, gab} NCA uses
the trapezoidal rule for measurement of area under the
concentration-time curve (AUC), and requires fewer assumptions than
model-based analysis. \citep{gab} NCA allows for estimation of various
PK parameters such as AUC, peak observed drug concentration
(C\textsubscript{max}), time of peak concentration
(T\textsubscript{max}), and elimination half-life. Particularly, AUC and
C\textsubscript{max} are often accepted as the criteria for approval of
bioequivalent drugs.

R, a widely-used computer language, is a suite of libraries of
statistical and mathematical computations. \citep{R-base} Despite its
relatively small base system compared with other commercial softwares
for NCA such as WinNonlin®{[}4{]} and Kinetica,{[}5{]} R has robust
functions for scientific computation and numerous add-in packages for
use in various fields. {[}6{]} Therefore, many efforts are being made to
replace commercial softwares with R packages.

In this article, we introduce two newly-developed R packages, NonCompart
\citep{R-NonCompart} and ncar \citep{R-ncar}, that are compatible with
SDTM (Study Data Tabulation Model)-formatted dataset of CDISC (Clinical
Data Interchange Standards Consortium), which is the standard of
documentation submitted to regulatory authorities,{[}7{]} while
providing a practical method for producing complete NCA reports.

\chapter{Methods}\label{methods}

\section{R packages (NonCompart and
ncar)}\label{r-packages-noncompart-and-ncar}

Two R packages (NonCompart and ncar) for NCA were developed in the
open-source R programming language in order to allow free public use. R
packages can be installed and loaded using the following scripts.

\begin{Shaded}
\begin{Highlighting}[]
\KeywordTok{install.packages}\NormalTok{(}\KeywordTok{c}\NormalTok{(}\StringTok{'NonCompart'}\NormalTok{, }\StringTok{'ncar'}\NormalTok{))}
\KeywordTok{library}\NormalTok{(NonCompart)}
\KeywordTok{library}\NormalTok{(ncar)}
\end{Highlighting}
\end{Shaded}

Detailed documentation and examples for each package can be found on the
online user manual in the CRAN repository
(\url{http://cran.r-project.org/web/packages/NonCompart/index.html},
\url{http://cran.r-project.org/web/packages/ncar/index.html}) or
directly within the R console by entering ?function
(e.g.?NonCompart,?ncar ). These two packages are implemented in R and
can accept a set of input arguments that allow for generation of NCA
output. The names of most NCA metrics estimated by the function of these
packages are consistent with those used in WinNonlin®. (Table
\ref{tab:table1})

\begin{table}

\caption{\label{tab:table1}Description of PK parameters of WinNonlin and the R packages}
\centering
\begin{tabular}[t]{l|l|l}
\hline
Parameter & WinNonlin & Description\\
\hline
b0 & b0 & Intercept\\
\hline
CMAX & Cmax & Max Concentration (Conc)\\
\hline
CMAXD & Cmax\_D & Max Conc Norm by Dose\\
\hline
TMAX & Tmax\_D & Time of Cmax\\
\hline
TLAG & Tlag & Time Until First Nonzero Conc\\
\hline
CLST & Clast & Last Nonzero Conc\\
\hline
CLSTP & Clast\_pred & Last Nonzero Conc Pred\\
\hline
TLST & Tlast & Time of Last Nonzero Conc\\
\hline
LAMZHL & HL\_Lambda\_z & Half-Life Lambda z\\
\hline
LAMZ & Lambda\_z & Lambda z\\
\hline
LAMZLL & Lambda\_z & lower Lambda z Lower Limit\\
\hline
LAMZUL & Lambda\_z & upper Lambda z Upper Limit\\
\hline
LAMZNPT & No\_points\_Lambda\_z & Number of Points for Lambda z\\
\hline
CORRXY & Corr\_XY & Correlation Between Time X and Log Conc Y\\
\hline
R2 & Rsq & R Squared\\
\hline
R2ADJ & Rsq\_adjusted & R Squared Adjusted\\
\hline
AUCLST & AUClast & AUC to Last Nonzero Conc\\
\hline
AUCALL & AUCall & AUC All\\
\hline
AUCIFO & AUCINF\_obs & AUC Infinity Obs\\
\hline
AUCIFOD & AUCINF\_D\_obs & AUC Infinity Obs Norm by Dose\\
\hline
AUCIFP & AUCINF\_Pred & AUC Infinity Pred\\
\hline
AUCIFPD & AUCINF\_D\_pred & AUC Infinity Pred Norm by Dose\\
\hline
AUCPEO & AUC\_Extrap\_obs & AUC \%Extrapolation Obs\\
\hline
AUCPEP & AUC\_Extrap\_pred & AUC \%Extrapolation Pred\\
\hline
AUMCLST & AUMClast & AUMC to Last Nonzero Conc\\
\hline
AUMCIFO & AUMCINF\_obs & AUMC Infinity Obs\\
\hline
AUMCIFP & AUMCINF\_pred & AUMC Infinity Pred\\
\hline
AUMCPEO & AUMC\_Extrap\_obs & AUMC \%Extrapolation Obs\\
\hline
AUMCPEP & AUMC\_Extrap\_pred & AUMC \% Extrapolation Pred\\
\hline
VZFO & Vz\_F\_obs & Vz Obs by F\\
\hline
VZFP & Vz\_F\_p & Vz Pred by F\\
\hline
CLFO & Cl\_F\_obs & Total CL Obs by F\\
\hline
CLFP & Cl\_F\_pred & Total CL Pred by F\\
\hline
MRTEVLST & MRTlast & MRT Extravasc to Last Nonzero Conc\\
\hline
MRTEVIFO & MRTINF\_obs & MRT Extravasc Infinity Obs\\
\hline
MRTEVIFP & MRTINF\_pred & MRT Extravasc Infinity Pred\\
\hline
\end{tabular}
\end{table}

\section{Software}\label{software}

WinNonlin® (Pharsight, Mountain View, CA, USA){[}4{]} in
Microsoft-Windows 7 (64 bit) was used for the computation. R 3.4.2 in
Microsoft-Windows 7 (64 bit) was used for the comparison of calculated
values.

\section{Dataset}\label{dataset}

To compare the outputs generated by R packages and WinNonlin®, we used
Theoph dataset obtained from the R software. The Theoph dataset has 132
observations from 12 subjects. A portion of the Theoph dataset (subject
ID = 8) is shown in Table \ref{tab:table2}.

\begin{table}

\caption{\label{tab:table2}Part of Theoph dataset with information on key (subject),
time, and concentration}
\centering
\begin{tabular}[t]{r|r|r|r|r}
\hline
Subject & Weight (kg) & Dose (mg) & Time (h) & Concentration (mg/ml)\\
\hline
8 & 70.5 & 4.53 & 0.00 & 0.00\\
\hline
8 & 70.5 & 4.53 & 0.25 & 3.05\\
\hline
8 & 70.5 & 4.53 & 0.52 & 3.05\\
\hline
8 & 70.5 & 4.53 & 0.98 & 7.31\\
\hline
8 & 70.5 & 4.53 & 2.02 & 7.56\\
\hline
8 & 70.5 & 4.53 & 3.53 & 6.59\\
\hline
8 & 70.5 & 4.53 & 5.05 & 5.88\\
\hline
8 & 70.5 & 4.53 & 7.15 & 4.73\\
\hline
8 & 70.5 & 4.53 & 9.05 & 4.57\\
\hline
8 & 70.5 & 4.53 & 12.10 & 3.00\\
\hline
8 & 70.5 & 4.53 & 24.12 & 1.25\\
\hline
\end{tabular}
\end{table}

\chapter{Results}\label{results}

\section{NonCompart package: performance of
NCA}\label{noncompart-package-performance-of-nca}

This package conducts NCA as similarly as possible to the most widely
used commercial PK analysis software. The NonCompart package has two
main functions, tblNCA and sNCA, for use in multiple subjects and one
subject, respectively. Figure 1 shows an example of output by tblNCA.
The input data for tblNCA( ) should be in a long format as exemplified
by the Theoph dataset. It is possible to input several keys such as
subject demographics and information regarding dose, period, or
sequence; the result of tblNCA( ) will print the key columns and the
carried keys can be further used for additional statistical analysis
(i.e.~descriptive statistics, bioequivalence test, t-test, or ANOVA).
The adm argument can be `Extravascular', `Bolus', or `Infusion' and the
down argument can be either `Linear' or `Log'. The greatest advantage of
this package is that the outputs produced by this package are compatible
with those of pharmacokinetic parameter (PP) TESTCD of CDISC SDTM.

\begin{Shaded}
\begin{Highlighting}[]
\KeywordTok{tblNCA}\NormalTok{(Theoph, }\DataTypeTok{key =} \StringTok{"Subject"}\NormalTok{, }\DataTypeTok{colTime =} \StringTok{"Time"}\NormalTok{, }\DataTypeTok{colConc =} \StringTok{"conc"}\NormalTok{, }\DataTypeTok{dose =} \DecValTok{320}\NormalTok{, }
       \DataTypeTok{adm =} \StringTok{"Extravascular"}\NormalTok{, }\DataTypeTok{dur =} \DecValTok{0}\NormalTok{, }\DataTypeTok{doseUnit =} \StringTok{"mg"}\NormalTok{, }\DataTypeTok{timeUnit =} \StringTok{"h"}\NormalTok{, }\DataTypeTok{concUnit =} \StringTok{"mg/L"}\NormalTok{, }
       \DataTypeTok{down =} \StringTok{"Linear"}\NormalTok{)}
\end{Highlighting}
\end{Shaded}

IntAUC() function calculates interval (partial) AUC (from t 1 and t 2 )
with the given series of time and concentration. The interval AUC
(0.5--11 hour) of the subject 8 can be calculated using the Theoph
dataset with the following R script.

\begin{Shaded}
\begin{Highlighting}[]
\NormalTok{Time =}\StringTok{ }\NormalTok{Theoph[Theoph}\OperatorTok{$}\NormalTok{Subject }\OperatorTok{==}\StringTok{ }\DecValTok{8}\NormalTok{, }\StringTok{"Time"}\NormalTok{]}
\NormalTok{Concentration =}\StringTok{ }\NormalTok{Theoph[Theoph}\OperatorTok{$}\NormalTok{Subject }\OperatorTok{==}\StringTok{ }\DecValTok{8}\NormalTok{, }\StringTok{"conc"}\NormalTok{]}
\NormalTok{Res =}\StringTok{ }\KeywordTok{sNCA}\NormalTok{(Time, Concentration,}\DataTypeTok{dose =} \DecValTok{320}\NormalTok{, }\DataTypeTok{concUnit =} \StringTok{"mg/L"}\NormalTok{)}
\KeywordTok{IntAUC}\NormalTok{(Time, Concentration, }\DataTypeTok{t1 =} \FloatTok{0.5}\NormalTok{, }\DataTypeTok{t2 =} \DecValTok{11}\NormalTok{, Res)}
\end{Highlighting}
\end{Shaded}

\begin{verbatim}
## [1] 58.26022
\end{verbatim}

\section{ncar package: generation of NCA
reports}\label{ncar-package-generation-of-nca-reports}

This package generates complete NCA reports including plots with both
linear and logarithmic scale. Its two main functions are pdfNCA and
rtfNCA, which produce pdf file format and rtf file format, respectively.
The generated reports are similar to those generated from commercial
softwares, but like NonCompart, this package has the advantage of using
PPTESTCD of CDISC SDTM. ncar produces NCA reports through NonCompart and
converts them into Microsoft Word format when using rtfNCA( ), which is
convenient for editing. Re- ports generated by pdfNCA( )function show
individual plots with trend lines that joins the dots used for
calculating terminal slopes. Figure 2 shows an example of an NCA report
in pdf for- mat and an individual concentration-time plot.

\begin{Shaded}
\begin{Highlighting}[]
\KeywordTok{pdfNCA}\NormalTok{(}\DataTypeTok{fileName =} \StringTok{"pdfNCA-Theoph.pdf"}\NormalTok{,}
\NormalTok{Theoph, }\DataTypeTok{colSubj =} \StringTok{"Subject"}\NormalTok{, }\DataTypeTok{colTime =}
\StringTok{"Time"}\NormalTok{, }\DataTypeTok{colConc =} \StringTok{"conc"}\NormalTok{, }\DataTypeTok{dose =} \DecValTok{320}\NormalTok{,}
\DataTypeTok{doseUnit =} \StringTok{"mg"}\NormalTok{, }\DataTypeTok{timeUnit =} \StringTok{"h"}\NormalTok{, concU}\OperatorTok{-}
\DataTypeTok{nit =} \StringTok{"mg/L"}\NormalTok{, }\DataTypeTok{down =} \StringTok{"Linear"}\NormalTok{)}
\end{Highlighting}
\end{Shaded}

\section{Validation of NCA results between R packages and
WinNonlin®}\label{validation-of-nca-results-between-r-packages-and-winnonlin}

To demonstrate the accordance of outputs by ncar package and WinNonlin®,
we performed NCA using Theoph dataset ob- tained from the R software.
For comparison of the NCA results, we selected the following conditions:
extravascular, linear-up linear-down, and best fit. We found no
discrepancy between the two results as shown in Table \ref{tab:table2}
(a randomized subject, Subject ID = 8).

\begin{table}

\caption{\label{tab:table3}Comparison of NCA results generated from WinNonlin and ncar package}
\centering
\begin{tabular}[t]{l|l|l}
\hline
Parameter & WinNonlin & ncar\\
\hline
CMAX & 7.56 mg/L & 7.5600 mg/L\\
\hline
CMAXD & 0.023625 mg/L/mg & 0.0236 mg/L/mg\\
\hline
TMAX & 2.02 h & 2.0200 h\\
\hline
TLAG & 0 h & 0.0000 h\\
\hline
CLST & 1.25 mg/L & 1.2500 mg/L\\
\hline
TLST & 24.12 h & 24.1200 h\\
\hline
LAMZHL & 8.510037883 h & 8.5100 h\\
\hline
LAMZ & 0.08145054 /h & 0.0815 /h\\
\hline
LAMZLL & 3.53 h & 3.5300 h\\
\hline
LAMZUL & 24.12 h & 24.1200 h\\
\hline
LAMZNPT & 6 & 6\\
\hline
CORRXY & -0.995496053 & -0.9955\\
\hline
R2 & 0.991012391 & 0.991\\
\hline
R2ADJ & 0.988765489 & 0.9888\\
\hline
AUCLST & 88.55995 h*mg/L & 88.5600 h*mg/L\\
\hline
AUCALL & 88.55995 h*mg/L & 88.5600 h*mg/L\\
\hline
AUCIFO & 103.906687 h*mg/L & 103.9067 h*mg/L\\
\hline
AUCIFOD & 0.324708 h*mg/L/mg & 0.3247 h*mg/L/mg\\
\hline
AUCIFP & 103.643051 h*mg/L & 103.6431 h*mg/L\\
\hline
AUCIFP & 0.323884 h*mg/L/mg & 0.3239 h*mg/L/mg\\
\hline
AUCPEO & 14.77\% & 14.77\%\\
\hline
AUCPEP & 14.55\% & 14.55\%\\
\hline
AUMCLST & 739.534598 h2*mg/L & 739.5346 h2*mg/L\\
\hline
AUMCIFO & 1298.115755 h2*mg/L & 1298.1158 h2*mg/L\\
\hline
AUMCIFP & 1288.520116 h2*mg/L & 1288.5201 h2*mg/L\\
\hline
AUMCPEO & 43.03\% & 43.03\%\\
\hline
AUMCPEP & 42.61\% & 42.61\%\\
\hline
VZFO & 37.81050811 L & 37.8105 L\\
\hline
VZFP & 37.90668616 L & 37.9067 L\\
\hline
CLFO & 3.079686301 L/h & 3.0797 L/h\\
\hline
CLFP & 3.087520055 L/h & 3.0875 L/h\\
\hline
MRTEVLST & 8.35066639 h & 8.3507 h\\
\hline
MRTEVIFO & 12.49309159 h & 12.4931 h\\
\hline
MRTEVIFP & 12.43228656 h & 12.4323 h\\
\hline
\end{tabular}
\end{table}

In order to further validate these packages, we compared NCA results
using Indometh, another available dataset of the R software as well as
other datasets of a number of subjects from several phase 1 clinical
trials with different dosing routes such as infusion, bolus, and oral
route. As a result, we could not find any discrepancy between outputs
generated by the R packages and WinNonlin®.

\chapter{Discussion}\label{discussion}

We developed two R packages - NonCompart and ncar for NCA. Through these
packages, we aimed to imple ment the following functionalities for
performing NCA: 1) CDISC SDTM terms; 2) automatic slope selection with
the same criterion of WinNonlin®; 3) supporting both `linear-up
linear-down' and `linear-up log-down' method; and 4) interval (partial)
AUCs with `linear' or `log' interpolation method. These packages are
convenient and efficient because they enable prepara- tion of data and
NCA as well as generation of reports includ- ing plots together in R
software. As shown in Figure 2B, the NCA plot allows for automatic slope
selection, however, it is not possible to manually choose the points
used for calculating ter- minal slope. In addition, any error or change
can easily be fixed, and users may choose calculation methods between
linear and logarithmic, which support `linear-up linear-down' and
`linear- up log-down' method, respectively. Our results showed that our
R packages meet the aforementioned objectives. Since the PPTESTCD of
SDTM is used in the R packages, it is helpful to construct PP domain. In
the present practice, one has to change variables from WinNonlin® one by
one, which is an especially difficult task for those without specific
knowledge on SDTM. A number of packages can perform NCA, but no
package-even commercial softwares-can give outputs in the format of SDTM
or receive SDTM-formatted input data. It is important to ensure that the
reports are legible to sponsors and regulatory bodies by generating a
consistent and systematic re- sult, as well as the exact results of NCA.
As shown in Table 3, comparison of NCA results obtained by WinNonlin®
and ncar package (including another package) showed no significant
discrepancies. These two R packages are fast and easy-to-use tool-set
that can successfully perform NCA with concentration--time data.
Specifically, the ncar package can produce a comprehensive set of
graphical and tabular outputs that summarize the NCA results, which is a
complete report in pdf or rtf format. Our two newly-developed packages
are free and open-source, so they can be used to develop other useful
packages as well. We hope that NonCompart and ncar packages will enable
researchers to easily perform NCA, and contribute to facilitation of
drug discovery process.

\appendix \addcontentsline{toc}{chapter}{\appendixname}


\chapter{Environment}\label{environment}

\begin{tabular}{l|l}
\hline
Package & Version\\
\hline
tidyverse & 1.2.1\\
\hline
NonCompart & 0.3.3\\
\hline
ncar & 0.3.7\\
\hline
knitr & 1.20\\
\hline
\end{tabular}

\begin{verbatim}
##  setting  value                       
##  version  R version 3.4.3 (2017-11-30)
##  system   x86_64, mingw32             
##  ui       RTerm                       
##  language (EN)                        
##  collate  Korean_Korea.949            
##  tz       Asia/Seoul                  
##  date     2018-03-08                  
## 
##  package     * version    date       source                              
##  assertthat    0.2.0      2017-04-11 CRAN (R 3.4.0)                      
##  backports     1.1.2      2017-12-13 CRAN (R 3.4.3)                      
##  base        * 3.4.3      2017-11-30 local                               
##  bindr         0.1.0.9000 2018-02-08 Github (krlmlr/bindr@4b20179)       
##  bindrcpp      0.2.0.9000 2018-02-08 Github (krlmlr/bindrcpp@7553d4f)    
##  bookdown      0.7        2018-02-18 CRAN (R 3.4.3)                      
##  broom         0.4.3      2017-11-20 CRAN (R 3.4.2)                      
##  cellranger    1.1.0      2016-07-27 CRAN (R 3.4.0)                      
##  cli           1.0.0      2017-11-05 CRAN (R 3.4.2)                      
##  colorspace    1.3-2      2016-12-14 CRAN (R 3.4.0)                      
##  compiler      3.4.3      2017-11-30 local                               
##  crayon        1.3.4      2018-03-02 Github (gaborcsardi/crayon@95b3eae) 
##  datasets    * 3.4.3      2017-11-30 local                               
##  devtools      1.13.5     2018-02-18 CRAN (R 3.4.3)                      
##  digest        0.6.15     2018-01-28 CRAN (R 3.4.3)                      
##  dplyr       * 0.7.4.9000 2018-02-08 Github (tidyverse/dplyr@0a2c208)    
##  evaluate      0.10.1     2017-06-24 CRAN (R 3.4.1)                      
##  forcats     * 0.3.0      2018-02-19 CRAN (R 3.4.3)                      
##  foreign       0.8-69     2017-06-22 CRAN (R 3.4.3)                      
##  ggplot2     * 2.2.1      2016-12-30 CRAN (R 3.4.0)                      
##  glue          1.2.0      2017-10-29 CRAN (R 3.4.2)                      
##  graphics    * 3.4.3      2017-11-30 local                               
##  grDevices   * 3.4.3      2017-11-30 local                               
##  grid          3.4.3      2017-11-30 local                               
##  gtable        0.2.0      2016-02-26 CRAN (R 3.4.0)                      
##  haven         1.1.1      2018-01-18 CRAN (R 3.4.3)                      
##  hms           0.4.1      2018-01-24 CRAN (R 3.4.3)                      
##  htmltools     0.3.6      2017-04-28 CRAN (R 3.4.0)                      
##  httr          1.3.1      2017-08-20 CRAN (R 3.4.1)                      
##  jsonlite      1.5        2017-06-01 CRAN (R 3.4.0)                      
##  knitr       * 1.20       2018-02-20 CRAN (R 3.4.3)                      
##  lattice       0.20-35    2017-03-25 CRAN (R 3.4.3)                      
##  lazyeval      0.2.1      2017-10-29 CRAN (R 3.4.2)                      
##  lubridate     1.7.3      2018-02-27 CRAN (R 3.4.3)                      
##  magrittr      1.5        2014-11-22 CRAN (R 3.4.0)                      
##  memoise       1.1.0      2017-04-21 CRAN (R 3.4.0)                      
##  methods       3.4.3      2017-11-30 local                               
##  mnormt        1.5-5      2016-10-15 CRAN (R 3.4.0)                      
##  modelr        0.1.1      2017-07-24 CRAN (R 3.4.1)                      
##  munsell       0.4.3      2016-02-13 CRAN (R 3.4.0)                      
##  ncar        * 0.3.7      2017-08-16 CRAN (R 3.4.1)                      
##  nlme          3.1-131.1  2018-02-16 CRAN (R 3.4.3)                      
##  NonCompart  * 0.3.3      2017-08-16 CRAN (R 3.4.1)                      
##  parallel      3.4.3      2017-11-30 local                               
##  pillar        1.2.1      2018-02-27 CRAN (R 3.4.3)                      
##  pkgconfig     2.0.1      2017-03-21 CRAN (R 3.4.0)                      
##  plyr          1.8.4      2016-06-08 CRAN (R 3.4.0)                      
##  psych         1.7.8      2017-09-09 CRAN (R 3.4.1)                      
##  purrr       * 0.2.4.9000 2018-03-02 Github (tidyverse/purrr@84ce1ad)    
##  R.methodsS3   1.7.1      2016-02-16 CRAN (R 3.4.0)                      
##  R.oo          1.21.0     2016-11-01 CRAN (R 3.4.0)                      
##  R6            2.2.2      2017-06-17 CRAN (R 3.4.1)                      
##  Rcpp          0.12.15    2018-01-20 CRAN (R 3.4.3)                      
##  readr       * 1.1.1      2017-05-16 CRAN (R 3.4.0)                      
##  readxl        1.0.0      2017-04-18 CRAN (R 3.4.0)                      
##  reshape2      1.4.3      2017-12-11 CRAN (R 3.4.3)                      
##  rlang         0.2.0      2018-02-20 CRAN (R 3.4.3)                      
##  rmarkdown     1.9        2018-03-01 CRAN (R 3.4.3)                      
##  rprojroot     1.3-2      2018-01-03 CRAN (R 3.4.3)                      
##  rstudioapi    0.7        2017-09-07 CRAN (R 3.4.1)                      
##  rtf         * 0.4-11     2013-11-12 CRAN (R 3.4.0)                      
##  rvest         0.3.2      2016-06-17 CRAN (R 3.4.0)                      
##  scales        0.5.0      2017-08-24 CRAN (R 3.4.1)                      
##  stats       * 3.4.3      2017-11-30 local                               
##  stringi       1.1.6      2017-11-17 CRAN (R 3.4.2)                      
##  stringr     * 1.3.0      2018-02-19 CRAN (R 3.4.3)                      
##  tibble      * 1.4.2      2018-01-22 CRAN (R 3.4.3)                      
##  tidyr       * 0.8.0      2018-01-29 CRAN (R 3.4.3)                      
##  tidyselect    0.2.4      2018-02-26 CRAN (R 3.4.3)                      
##  tidyverse   * 1.2.1      2017-11-14 Github (tidyverse/tidyverse@3769ff2)
##  tools         3.4.3      2017-11-30 local                               
##  utils       * 3.4.3      2017-11-30 local                               
##  withr         2.1.1.9000 2018-03-02 Github (r-lib/withr@5d05571)        
##  xfun          0.1        2018-01-22 CRAN (R 3.4.3)                      
##  xml2          1.2.0      2018-01-24 CRAN (R 3.4.3)                      
##  yaml          2.1.17     2018-02-27 CRAN (R 3.4.3)
\end{verbatim}

\chapter{References}\label{references}

\begin{enumerate}
\def\labelenumi{\arabic{enumi}.}
\tightlist
\item
  Acharya C, Hooker AC, Turkyilmaz GY, Jonsson S, Karlsson MO. A diag-
  nostic tool for population models using non-compartmental analysis:
  The ncappc package for R. Comput Methods Programs Biomed 2016;127:
  83-93. doi: 10.1016/j.cmpb.2016.01.013.
\item
  Gabrielsson J, Weiner D. Non-compartmental analysis. Methods Mol Biol
  2012;929:377--389.
\item
  The R Project for Statistical Computing. R.
  \url{http://www.r-project.org/} Accessed 6 November 2017.
\item
  Kinetica. \url{http://www.kinetica.com/} Accessed 28 February 2018
\item
  Certara. Phenix WinNonlin®.
  \url{https://www.certara.com/software/pkpd-}
  modeling-and-simulation/phoenix-winnonlin/?ap\%5B0\%5D=PKPD/ Accessed
  28 February 2018.
\item
  Kim MG, Yim DS, Bae KS. R-based reproduction of the estimation process
  hidden behind NONMEN Part 1: first-order approximation method. Transl
  Clin Pharmacol 2015;23:1-7.
\item
  Study Data Tabulation Model Implementation Guide: Human Clinical
  Trials Version 3.2, Clinical Data Interchange Standards Consortium.
  \url{https://www}.
  cdisc.org/system/files/all/standard\_category/application/pdf/sdtmig\_
  v3.2.pdf. Accessed 28 February 2018.
\end{enumerate}

\bibliography{bib/packages.bib,bib/manual.bib}

\end{document}
